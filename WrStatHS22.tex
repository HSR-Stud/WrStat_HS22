% !TeX program = xelatex
% !TeX encoding = utf8
% !TeX root = WrStatHS22.tex

%% TODO: publish to CTAN
\documentclass[margin=normal]{tex/hsrzf}

%%%%%%%%%%%%%%%%%%%%%%%%%%%%%%%%%%%%%%%%%%%%%%%%%%%
% Packages

%% TODO: publish to CTAN
\usepackage{tex/hsrstud}

%% Language configuration
\usepackage{multicol}
\usepackage{polyglossia}
\usepackage{tikz}
\setdefaultlanguage[variant=swiss]{german}

%% License configuration
\usepackage[
    type={CC},
    modifier={by-nc-sa},
    version={4.0},
    lang={german},
]{doclicense}

%%%%%%%%%%%%%%%%%%%%%%%%%%%%%%%%%%%%%%%%%%%%%%%%%%%
% Metadata

\course{Elektrotechnik}
\module{WrStat}
\semester{Herbstsemester 2022}

\authoremail{joel.leirer@ost.ch}
\author{\textsl{Joël Leirer} -- \texttt{\theauthoremail}}

% did someone help you with this work?
\contributors{

}

\title{\texttt{\themodule} Zusammenfassung}
\date{\thesemester}

%%%%%%%%%%%%%%%%%%%%%%%%%%%%%%%%%%%%%%%%%%%%%%%%%%%
% Document

\begin{document}

% use roman numberals for introductiory pages
\pagenumbering{roman}

\maketitle

% \begin{abstract}
% \end{abstract}

% show the names of the people who contributed to this document.
% \section*{Contributors}
% \thecontributors


\section*{Lizenz}
\doclicenseThis

\clearpage

\tableofcontents

% actual content
\clearpage
\setcounter{page}{1}
\pagenumbering{arabic}

\section{Kombinatorik}
\subsection{Zählregeln}
\begin{tabular}{c c c}
    Disjunktive Vereinigung                & Schnittmengen & Produkt \\
    \begin{tikzpicture}
        \fill[red!30!white]   (0,0) circle (0.8);
        \fill[green!30!white] (1.7,0) circle (0.8);
        \node at (0,0)    {$A$};
        \node at (1.7,0)   {$B$};
    \end{tikzpicture}
                                           &
    \begin{tikzpicture}
        \begin{scope}[blend group = soft light]
            \fill[red!30!white]   ( 90:0.8) circle (0.8);
            \fill[green!30!white] (180:0.8) circle (0.8);
        \end{scope}
        \node at ( 90:0.8)  {$A$};
        \node at (180:0.8)   {$B$};
    \end{tikzpicture}
                                           &
    \begin{tikzpicture}
        \fill[black!30!white] (0.3,-0.3) rectangle (1.6,-1.6);
        \fill[red!30!white]  (0.3,0) rectangle (1.6,-0.25);
        \fill[green!30!white](0,-0.3) rectangle (0.25,-1.6);
        \node at (0.95, 0.2) {$A$};
        \node at (-0.2, -0.95) {$B$};
        \node at (0.95, -0.95) {$ A \times B $};
    \end{tikzpicture}
    \\\\
    $|A \cup B| = |A| + |B| $              &
    $|A \cup B| = |A| + |B| - |A \cap B| $ &
    $|A \times B| = |A| \cdot |B| $
\end{tabular}
\subsection{Kombinatorik - Regeln}
\begin{tabular}{|l| p{10cm}|}
    \hline Anordnung - Permutation &
    auf wie viele Arten kann man $n$ Objekte anordnen? \newline
    $ n \cdot (n-1) \cdot (n-2) \cdots \cdot 1 = n! $         \\
    \hline Auswahl - Kombination   &
    Auf wie viele Arten kann man k aus n Auswählen? \newline
    Auswahl: $ n \cdot (n-1) \cdot \cdots \cdot (n-(k+1)) = \frac{n!}{(n-k)!}$
    Anordnung der Auswahl:  $k \cdots (k-1) \cdot \cdots \cdot = k!$
    Insgesamt: $ C^n_k = \frac{n!}{k!(n-k)!} = \binom{n}{k} $ \\
    \hline Variation               &
    Perlenkettenproblem:
    auf wie viele Arten kann man eine Perlenkette der Länge k aus n Farben herstellen? \newline
    $V_n,k = n \cdot n^k-1 = n^k$                             \\
    \hline Rekursion               & /TODO:
    \\
    \hline Erzeugende Funktion     & /TODO:
    \\
    \hline
\end{tabular}

\section{Ereignisse und Wahrscheinlichkeit}
\subsection{Ereignisse}
\begin{multicols}{2}
    \begin{tikzpicture}
        \begin{scope}[blend group = soft light]
            \fill [red!30!white] (0,0) circle [x radius = 1.2, y radius = 1.8];
            \fill [green!30!white] (2,-1) circle [y radius = 1.2, x radius = 1.8];
        \end{scope}
        \node [red] at (0.0, 1) {$A$};
        \node [green] at (3, -0.95) {$B$};
        \node [black]at (0, 0.2) {$ \omega $};
        \node [black]at  (3.5, 1.5) {$ \Omega $};
        \fill [black] (0,0) circle [radius = 0.05];
        \draw [black] (-1.5,2) rectangle (4,-2.5);
    \end{tikzpicture}
    \begin{itemize}
        \item $\omega$ = Elementarereignis (Versuchsausgang)
        \item $\Omega$ = Menge der Elementarereignisse
        \item $A$ und $B$ = Teilmengen von $\Omega$ (hier $\omega \in A, \omega \notin B$)
        \item Spezialfälle:
              \begin{itemize}
                  \item $A =\Omega \subset \Omega$: A tritt immer ein (sicheres Ereignis)
                  \item $B = \emptyset \subset \Omega$: B tritt nie ein (unmögliches Ereignis)
              \end{itemize}
    \end{itemize}
\end{multicols}
\begin{tabular}{l m{4cm}}
    \begin{tabular}{|p{2.5cm} |l|}
        \hline
        Modell                          & Begriff                                      \\
        \hline
        $\omega$                        & Versuchsausgang, Elementarereignis           \\
        $\Omega$                        & alle Versuchsausgänge                        \\
        $A \subset \Omega$              & Ereignis                                     \\
        $\omega \in A$                  & Ereignis ist eingetreten                     \\
        $\Omega$                        & sicheres Ereignis, tritt immer ein           \\
        $\emptyset$                     & unmögliches Ereignis, kann nicht eingetreten \\
        $A \cap B$                      & A und B treten ein                           \\
        $A \cup B$                      & A oder B treten ein                          \\
        $A \subset B$                   & A hat B zur Folge                            \\
        $\bar{A} = \Omega \setminus A $ & nicht A                                      \\
        \hline
    \end{tabular}
     &
    Rechenregeln:
    \begin{itemize}
        \item $A \cap ( B \cup C) = (A \cap B ) \cup (A \cap C) $
        \item $A \cup ( B \cap C) = (A \cup B ) \cap (A \cup C) $
        \item $\overline{A \cap B} = \overline{A}  \cup \overline{B}$
        \item  $\overline{A \cup B} = \overline{A} \cap \overline{B}$
    \end{itemize}
\end{tabular}
\subsection{Wahrscheinlichkeit}
\begin{tabular}{m{7.5cm} m{12cm}}
\subsubsection*{Axiome eines Wahrscheinlichkeitsraumes}
\begin{itemize}
    \item Wertebereich: \newline $0 \leq P(A) \leq 1$
    \item Wahrscheinlichkeit sicheres Ereignis: \newline$P(\Omega) = 1$
    \item Disjunktive Vereinigung:\newline $P(A_1 \cup A_2 \cup \dots \cup A_n \cup \dots)
              \newline = P(A_1) + P(A_2) + \dots + P(A_n) + \dots$
\end{itemize}
&
\subsubsection*{Eigenschaften der Wahrscheinlichkeit}


    \begin{itemize}
        \item Wahrscheinlichkeit des unmöglichen Ereignisses \newline $P(\emptyset) = 0$ \newline
              Auch $\neq \emptyset$ Ereignisse können Wahrscheinlichkeit 0 haben!
        \item Wahrscheinlichkeit des komplementären Ereignisses \newline
              $P(\overline{A}) = P(\Omega \setminus A) = 1 - P(A)$
        \item Wahrscheinlichkeit der Differenz zweier Ereignisse \newline
              $P(A \setminus B) = P(A) - P(A \cap B)$
        \item Wahrscheinlichkeit der Vereinigung zweier Ereignisse \newline
              $P(A \cup B) = P(A) + P(B) - P(A \cap B)$

    \end{itemize}
\end{tabular}
\subsection{Bedingte Wahrscheinlichkeit}
\subsubsection*{Satz von Bayes}
$P(R|T) \cdot P(T) = P(R \cap T) = P(T|R) \cdot P(R)$
\\ Daraus folgt: $P(R|T) = P(T|R) \frac{P(R)}{P(T)}$ 
und $P(T|R) = P(R|T) \frac{P(T)}{P(R)}$ 
\subsection*{Note}
\begin{itemize}
    \item Laplace-Experiment: Experiment bei dem jedes Elementarereignis gleich wahrscheinlich ist.
    \item Bernoulli-Experiment: Experiment mit zwei Versuchsausgängen mit Wahrscheinlichkeiten p und 1-p.
\end{itemize}

\end{document}
