% !TeX program = xelatex
% !TeX encoding = utf8
% !TeX root = WrStatHS22.tex

%% TODO: publish to CTAN
\documentclass[margin=normal]{tex/hsrzf}

%%%%%%%%%%%%%%%%%%%%%%%%%%%%%%%%%%%%%%%%%%%%%%%%%%%
% Packages

%% TODO: publish to CTAN
\usepackage{tex/hsrstud}

%% Language configuration
\usepackage{multicol}
\usepackage{polyglossia}
\usepackage{tikz}
\usetikzlibrary{shapes.geometric}
\usepackage{pgfplots}
\usepackage{pgfplotstable}
\pgfplotsset{compat = newest}
\setdefaultlanguage[variant=swiss]{german}
\usepackage{color, colortbl}
\usepackage{graphicx}
\usepackage{pdfpages}


%% Colors ant Tables
\newcolumntype{g}{>{\columncolor{Gray}}c}
\definecolor{Gray}{gray}{0.8}
\definecolor{LightGray}{gray}{0.9}

%% License configuration
\usepackage[
    type={CC},
    modifier={by-nc-sa},
    version={4.0},
    lang={german},
]{doclicense}

%%%%%%%%%%%%%%%%%%%%%%%%%%%%%%%%%%%%%%%%%%%%%%%%%%%
% Metadata

\course{Elektrotechnik}
\module{WrStat}
\semester{Herbstsemester 2022}

\authoremail{joel.leirer@ost.ch}
\author{\textsl{Joël Leirer} -- \texttt{\theauthoremail}}

% did someone help you with this work?
\contributors{

}

\title{\texttt{\themodule} Zusammenfassung}
\date{\thesemester}

%%%%%%%%%%%%%%%%%%%%%%%%%%%%%%%%%%%%%%%%%%%%%%%%%%%
% Document

\begin{document}

% use roman numberals for introductiory pages
\pagenumbering{roman}

\maketitle

% \begin{abstract}
% \end{abstract}

% show the names of the people who contributed to this document.
% \section*{Contributors}
% \thecontributors


\section*{Lizenz}
\doclicenseThis

\clearpage

\tableofcontents

% actual content
\clearpage
\setcounter{page}{1}
\pagenumbering{arabic}

\section{Kombinatorik}
\subsection{Zählregeln}
\begin{tabular}{c c c}
    Disjunktive Vereinigung                & Schnittmengen & Produkt \\
    \begin{tikzpicture}
        \fill[red!30!white]   (0,0) circle (0.8);
        \fill[green!30!white] (1.7,0) circle (0.8);
        \node at (0,0)    {$A$};
        \node at (1.7,0)   {$B$};
    \end{tikzpicture}
                                           &
    \begin{tikzpicture}
        \begin{scope}[blend group = soft light]
            \fill[red!30!white]   ( 90:0.8) circle (0.8);
            \fill[green!30!white] (180:0.8) circle (0.8);
        \end{scope}
        \node at ( 90:0.8)  {$A$};
        \node at (180:0.8)   {$B$};
    \end{tikzpicture}
                                           &
    \begin{tikzpicture}
        \fill[black!30!white] (0.3,-0.3) rectangle (1.6,-1.6);
        \fill[red!30!white]  (0.3,0) rectangle (1.6,-0.25);
        \fill[green!30!white](0,-0.3) rectangle (0.25,-1.6);
        \node at (0.95, 0.2) {$A$};
        \node at (-0.2, -0.95) {$B$};
        \node at (0.95, -0.95) {$ A \times B $};
    \end{tikzpicture}
    \\\\
    $|A \cup B| = |A| + |B| $              &
    $|A \cup B| = |A| + |B| - |A \cap B| $ &
    $|A \times B| = |A| \cdot |B| $
\end{tabular}
\subsection{Kombinatorik - Regeln}
\begin{tabular}{|l| p{10cm}|}
    \hline Anordnung - Permutation &
    auf wie viele Arten kann man $n$ Objekte anordnen? \newline
    $ n \cdot (n-1) \cdot (n-2) \cdots \cdot 1 = n! $         \\
    \hline Auswahl - Kombination   &
    Auf wie viele Arten kann man k aus n Auswählen? \newline
    Auswahl: $ n \cdot (n-1) \cdot \cdots \cdot (n-(k+1)) = \frac{n!}{(n-k)!}$
    Anordnung der Auswahl:  $k \cdots (k-1) \cdot \cdots \cdot = k!$
    Insgesamt: $ C^n_k = \frac{n!}{k!(n-k)!} = \binom{n}{k} $ \\
    \hline Variation               &
    Perlenkettenproblem:
    auf wie viele Arten kann man eine Perlenkette der Länge k aus n Farben herstellen? \newline
    $V_{n,k} = n^k$                                           \\
    \hline Erzeugende Funktion     &
    $f(z) = \sum \limits _{n = 0} ^{\infty} a_n \cdot z^n
        = a_0 + a_1 \cdot z + a_2 \cdot z^2 + \dots + a_n \cdot z^n$
    \\
    \hline
\end{tabular}

\section{Ereignisse und Wahrscheinlichkeit}
\subsection{Ereignisse}
\begin{multicols}{2}
    \begin{tikzpicture}
        \begin{scope}[blend group = soft light]
            \fill [red!30!white] (0,0) circle [x radius = 1.2, y radius = 1.8];
            \fill [green!30!white] (2,-1) circle [y radius = 1.2, x radius = 1.8];
        \end{scope}
        \node [red] at (0.0, 1) {$A$};
        \node [green] at (3, -0.95) {$B$};
        \node [black]at (0, 0.2) {$ \omega $};
        \node [black]at  (3.5, 1.5) {$ \Omega $};
        \fill [black] (0,0) circle [radius = 0.05];
        \draw [black] (-1.5,2) rectangle (4,-2.5);
    \end{tikzpicture}
    \begin{itemize}
        \item $\omega$ = Elementarereignis (Versuchsausgang)
        \item $\Omega$ = Menge der Elementarereignisse
        \item $A$ und $B$ = Teilmengen von $\Omega$ (hier $\omega \in A, \omega \notin B$)
        \item Spezialfälle:
              \begin{itemize}
                  \item $A =\Omega \subset \Omega$: A tritt immer ein
                  \item $B = \emptyset \subset \Omega$: B tritt nie ein
              \end{itemize}
    \end{itemize}
\end{multicols}
\begin{tabular}{l m{4cm}}
    \begin{tabular}{|p{2.5cm} |l|}
        \hline
        Modell                          & Begriff                                    \\
        \hline
        $\omega$                        & Versuchsausgang, Elementarereignis         \\
        $\Omega$                        & alle Versuchsausgänge                      \\
        $A \subset \Omega$              & Ereignis                                   \\
        $\omega \in A$                  & Ereignis ist eingetreten                   \\
        $\Omega$                        & sicheres Ereignis, tritt immer ein         \\
        $\emptyset$                     & unmögliches Ereignis, kann nicht eintreten \\
        $A \cap B$                      & A und B treten ein                         \\
        $A \cup B$                      & A oder B treten ein                        \\
        $A \subset B$                   & A hat B zur Folge                          \\
        $\bar{A} = \Omega \setminus A $ & nicht A                                    \\
        \hline
    \end{tabular}
     &
    Rechenregeln:
    \newline \resizebox{!}{8pt}{$A \cap ( B \cup C) = (A \cap B ) \cup (A \cap C) $}
    \newline \resizebox{!}{8pt}{$A \cup ( B \cap C) = (A \cup B ) \cap (A \cup C) $}
    \newline \resizebox{!}{9pt}{$\overline{A \cap B} = \overline{A}  \cup \overline{B}$}
    \newline \resizebox{!}{9pt}{$\overline{A \cup B} = \overline{A} \cap \overline{B}$}
    
\end{tabular}
\subsection{Wahrscheinlichkeit}
\begin{tabular}{m{7.5cm} m{12cm}}
    \subsubsection*{Axiome eines Wahrscheinlichkeitsraumes}
    \begin{itemize}
        \item Wertebereich: \newline $0 \leq P(A) \leq 1$
        \item Wahrscheinlichkeit sicheres Ereignis: \newline$P(\Omega) = 1$
        \item Disjunktive Vereinigung:\newline $P(A_1 \cup A_2 \cup \dots \cup A_n \cup \dots)
                  \newline = P(A_1) + P(A_2) + \dots + P(A_n) + \dots$
    \end{itemize}
     &
    \subsubsection*{Eigenschaften der Wahrscheinlichkeit}


    \begin{itemize}
        \item Wahrscheinlichkeit des unmöglichen Ereignisses \newline $P(\emptyset) = 0$ \newline
              Auch Ereignisse $\neq \emptyset$ können Wahrscheinlichkeit 0 haben!
        \item Wahrscheinlichkeit des komplementären Ereignisses \newline
              $P(\overline{A}) = P(\Omega \setminus A) = 1 - P(A)$
        \item Wahrscheinlichkeit der Differenz zweier Ereignisse \newline
              $P(A \setminus B) = P(A) - P(A \cap B)$
        \item Wahrscheinlichkeit der Vereinigung zweier Ereignisse \newline
              $P(A \cup B) = P(A) + P(B) - P(A \cap B)$
    \end{itemize}
\end{tabular}
\begin{multicols}{2}

    \subsubsection{Bedingte Wahrscheinlichkeit}
    Wahrscheinlichkeit das Ereignis $P(A)$ eintrifft,
    wenn Ereignis $P(B)$ eingetroffen ist: $P(A|B)$
    $$P(A|B)\frac{P(A\cap B)}{P(B)}$$
    wenn A und B unabhängig:
    $$P(A \cap B) = P(A) \cdot P(B)$$
    \subsubsection*{Satz von Bayes ("Bedingungsumkehrung")}
    $$P(R|T) \cdot P(T) = P(R \cap T) = P(T|R) \cdot P(R)$$
    $$P(R|T) = P(T|R) \frac{P(R)}{P(T)}$$
    \subsubsection{Totale Wahrscheinlichkeit}
    Sind nur Bedingte Wahrscheinlichkeiten $P(A|B)$ und die bedingende Wahrscheinlichkeit
    $P(B)$ bekannt, ergibt sich die totale Wahrscheinlichkeit $P(A)$ aus:
    $$P(A) = P(A|B) \cdot P(B) + P(A|\bar{B})\cdot P(\bar{B})$$
    \subsection*{Note}
    \begin{itemize}
        \item Laplace-Experiment: Experiment bei dem jedes Elementarereignis gleich wahrscheinlich ist.
        \item Bernoulli-Experiment: Experiment mit zwei Versuchsausgängen mit Wahrscheinlichkeiten p und 1-p.
    \end{itemize}
\end{multicols}

\section{Statistik}
\begin{multicols}{2}

    \subsection{Begriffe}
    \subsubsection*{Arithmetischer Mittelwert:}
    $$\bar{x} = \frac{x_0 + \dots + x_1}{n} = \frac{1}{n} \sum_{i=1}^n x_i $$
    \subsubsection*{Gewichteter Mittelwert:}
    $$\bar{x} = \frac{w_1x_1 + w_2x_2 + \dots + w_nx_n}{w_1 + w_2 + \dots + w_n}
        = \frac{\sum _{k=1}^n W_kx_k}{\sum_{k=1}^n w_k}$$
    Eigenschaften:
    $$y_i = ax_i + b \Rightarrow \bar{y} = a\bar{x} + b$$

    \subsubsection*{Geometrisches Mittel}
    $$G= \sqrt[n]{x_1\cdot x_2\cdot\dots\cdot x_n} = \sqrt[n]{\prod x_i}$$

    \subsubsection*{Median}
    Median ist ein Wert einer Liste der in der Mitte steht wenn sie der Grösse nach sortiert wird.
    Bei einer geraden Anzahl Werte ist er das Arithmetische Mittel der beiden mittigsten Werte.
    Bsp: $x = {74, 4, 8, 2, 1355, 6643} \Rightarrow x_{0.5} = 41$ \\
    Eigenschaften:
    $$y_i = ax_i + b \Rightarrow med(y_i) = a \cdot med(x_i) + b$$
    $$ med(x_i) = min(\sum _{i=1} ^n |x_i - \bar{x}|)$$

    \subsubsection*{Quantil}
    Eine Unterteilung der Beobachtungswerte bzw. der Fläche der Dichtekurve.
    $p$-Quantil $p \in \mathbb{R}$ links davon ist der Anteil $p$ in \% aller Zufallswerte bzw. Der Dichtekurve.
\end{multicols}



\subsection{Diskrete Zufallsvariable}
\begin{multicols}{2}

    \noindent Eine Zufallsvariable X ist eine Funktion die einem
    Versuchsausgang $\omega$ einen Wert $X(\omega)$ zuordnet.
    Sie ist \textbf{diskret} wenn sie einzelne genaue Zahlenwerte annehmen kann.
    Wenn sie beliebige Werte annehmen kann ist sie \textbf{stetig}.

    \subsubsection*{Erwartungswert}
    Der Erwartungswert ist der Wert der Zufallsvariable multipliziert mit ihrer Wahrscheinlichkeit.
    $$E(x) = \sum Wert \cdot Wahrscheinlichkeit$$
    $$ =\sum _{\omega \in \Omega} X(\omega) \cdot P({\omega}) = \sum _{i=1} ^{k} X(A_i) \cdot P(A_i)$$

    \subsubsection*{Rechenregeln}
    $$E(X+Y) = E(X) + E(Y)$$
    $$E(\lambda X) = \lambda E(X)$$

    \begin{itemize}
        \item Vorzeichen von XY unkorreliert \newline $\Rightarrow E(XY) = E(X)E(Y)$
        \item Unabhängig: $XY \geq 0 \Rightarrow E(XY) \geq 0$ \newline oder $XY\leq 0 \Rightarrow E(XY) \leq 0$
        \item X und Y sind unabhängig, wenn die Ereignisse x und y unabhängig sind.
    \end{itemize}
    \subsubsection*{Empirischer Erwartungswert}
    \resizebox{0.5\textwidth}{!}{$E(X) = \sum _i X(i) \cdot P(X=x_i) \approx \sum _i x_i \frac{n_i}{n} = \frac{x_1n_1 + \dots + x_in_i}{n}$}


\end{multicols}

\subsection{Varianz - Streumass}
\begin{multicols}{2}
    \subsubsection*{Definition}
    \noindent Die Varianz ist die Erwartete Abweichung vom Erwartungswert.
    Sie misst die Streuung einer Zufallsvariable. Sie ist gleich der Quadratischen
    Standardabweichung.
    $$var(X) = \sigma^2(X)$$

    $$var(x) = E((|X-E(X)|)^2) = E(X^2) - E(X)^2$$

    wenn X und Y \textbf{unabhängig} (oder unkorreliert):
    $$var(\lambda X) = \lambda^2 var(x)$$
    $$var(X+Y) = var(X) + var(Y)$$

    \subsubsection*{Satz von Tschebyscheff}
    Wenn X eine Zufallsvariable mir Erwartungswert $\mu = E(X)$ ist, dann gilt:
    $$P(|X-\mu|)\leq \frac{var(X)}{\epsilon^2}$$
    für $X-\mu = A$ und $X-\mu > \epsilon$
    \\[10pt]

    \subsubsection*{Satz von Bernoulli}
    \textbf{Stichprobe:}
    \\$X_1 \dots X_n$ sind Zufallsvariablen mit gleicher Varianz und und gleichem Erwartungswert.
        \\ \textbf{Mittelwert:}
        \\ \resizebox{.5\textwidth}{!}{$M_n = \frac{X_1+\dots+ X_n}{n}$, $E(M_n) = E(X), var(M_n) =\frac{1}{n} var(X)$}
        \\ \textbf{Satz:}
        \\$$P(|M_n -\mu|) \leq \frac{var(X)}{n\epsilon^2} $$
        für $|M_n -\mu| > \epsilon$

        \subsubsection*{Gesetz grosser Zahlen}
        Wenn die Wahrscheinlichkeit von Ereignis A gleich dem Erwartungswert ($P(A) = E(X)$), gilt:
        \\ \textbf{Relative Häufigkeit:}
        \\  \resizebox{.5\textwidth}{!}{$h_n = \frac{X_1+ \dots + X_n}{n}$, $E(h_n) = P(A)$ , $ var(h_n) = \frac{1}{n} var(X) = \frac{1}{n}P(A)(1-P(A))$}
        \\ \textbf{Satz von Bernoulli:}
        $$P(|h_n - P(A)|) \leq \frac{P(A)(1-P(A))}{n\epsilon^2} \leq \frac{1}{4n\epsilon^2} $$
        für $|h_n - P(A)| > \epsilon$

    \tiny{Je grösser die Anzahl Durchführungen, desto kleine die Wahrscheinlichkeit, dass die relative Häufigkeit stark von der Wahrscheinlichkeit abweicht.}
\end{multicols}

\newpage
\subsection{Lineare Regression}
\begin{multicols}{2}

    \begin{tikzpicture}
        \begin{axis}[
                xmin = 0, xmax = 11,
                ymin = 0, ymax = 11,
                width = 0.45\textwidth,
                height = 0.35\textwidth,
                xtick distance = 1,
                ytick distance = 1,
                grid = both,
                minor tick num = 1,
                major grid style = {lightgray},
                minor grid style = {lightgray!25},
                xlabel = {X ($x$)},
                ylabel = {Y ($y$)},
                legend cell align = {left},
                legend pos = north west
            ]
            % Plot data
            \addplot[
                teal,
                only marks
            ] table[x = t, y = x] {LinearReg.dat};
            % Linear regression
            \addplot[
                thick,
                orange
            ] table[
                    x = t,
                    y = {create col/linear regression={y=x}}
                ] {LinearReg.dat};
            % Add legend
            \addlegendentry{Daten}
            \addlegendentry{Lineare Regression};
        \end{axis}
    \end{tikzpicture}

    \noindent Linearer Zusammenhang zwischen Zufallsvariablen X und Y,     wobei, a und b zu Bestimmen sind, damit der Fehler möglichst klein wird:
    $$ Y = aX + b + Fehler$$
    \\ \textbf{Gleichungssystem:}
    \begin{eqnarray*}E(X^2)a + E(X)b& =& E(XY) \\ E(X)a + b & = & E(Y)\end{eqnarray*}
    daraus folgt:
    $$a= \frac{E(XY)- E(X)E(Y)}{E(X^2) - E(X)^2} = \frac{cov(X,Y)}{var(X)}$$
    $$b= E(Y) - aE(X)$$

    \noindent\textbf{Covarianz}\\
    $$cov(X,Y) = E(XY) - E(X)E(Y)$$

    \noindent\textbf{Fehlerberechnung:}
    $$\frac{var(Y-aX-b)}{var(Y)} = 1- \frac{cov(X,Y)^2}{var(X)var(Y)} = 1 -r^2 $$

    \noindent\textbf{Regressionskoeffizient:}
    $$r = \frac{cov(X,Y)}{\sqrt{var(X) \cdot var(Y)}} = \frac{E(XY) - E(X)E(Y)}{\sqrt{var(X) \cdot var(Y)}}$$
    \noindent\textbf{Bewertung des Modells} \\
    wenn $r^2 \simeq  1$ Passt das Modell gut.
\end{multicols}

\subsection{Stetige Zufallsvariable}
\begin{multicols}{2}

    \subsubsection*{Verteilungsfunktion}
    Die Verteilungsfunktion beschreib die Wahrscheinlichkeiten der Werte einer Zufallsvariable.
    $$F(x)= F_X(x) = P(X \leq x)$$
    Eigenschaften:
    \begin{itemize}
        \item monoton wachsend
        \item $0 \leq F_X(x) \leq 1$
        \item $\lim_{x \to -\infty}  F(x) = 0$
        \item $lim_{x \to \infty} F(x) = 1$
        \item $P(a < X \leq b) = F_x(b) - F_X(a)$
    \end{itemize}

    \subsubsection*{Wahrscheinlichkeitsdichte:}
    $$F_X(x) = \int \limits _{-\infty} ^{x} \varphi(\xi)d\xi$$

    Eigenschaften:
    \begin{itemize}
        \item $\varphi(x) \geq 0$
        \item $\int \limits _{-\infty} ^{\infty} \varphi(x)dx = 1$
        \item $P(a < X \leq b) =\int \limits _a ^b \varphi_X(x)dx$
    \end{itemize}

    \subsubsection{Erwartungswert:}
    $$E(X) = \sum x_i \cdot p(x_i) = \int \limits _{-\infty} ^{\infty} x \cdot \varphi(x)dx $$
    $$E(X^2) = \int \limits _{-\infty} ^{\infty} x^2 \cdot \varphi(x) dx$$
    \subsubsection*{Summe Stetiger Zufallsvariablen}
    Wenn $X$ und $Y$ unabhängig sind, dann hat die Zufallsvariable $Z = X + Y$ die Wahrscheinlichkeitsdichte:
    $$\varphi_Z(z) = \varphi_X * \varphi_Y(z) = \int \limits _{-\infty} ^{\infty}  \varphi_X(x) \varphi_Y(z-x)dx$$

    \subsubsection*{Summe von Gleichverteilungen}
    $$\varphi_X(x) = \varphi_Y(x)$$
    $$\varphi_{X+Y}(x) = \varphi_X * \varphi_Y(x) = \int \limits _{-\infty} ^{\infty} \varphi(y)\cdot \varphi(x-y)dy $$
\end{multicols}

\subsection{Verteilungen}
\begin{minipage}{0.5\textwidth}
    \subsubsection{Normalverteilung}
    X heisst Normalverteilt wenn gilt:
    $$\varphi(x) = \frac{1}{\sigma \sqrt{2\pi}} e^{-\frac{(x-\mu)^2}{2\sigma^2}}$$
    \subsubsection*{Standardisierung}
    Wenn X normalverteilt mit $E(X) = \mu$ und  \newline $var(X) = \sigma^2$ dann gilt:
    $$Z=\frac{X-\mu}{\sigma}$$
    Z ist die Standardnormalverteilung.
    \subsubsection*{Anwendung}
    \begin{enumerate}
        \item  Grundformel: $P(X<a) = P(\frac{X-\mu}{\sigma} \leq \frac{a-\mu}{\sigma})$
        \item  da P(X) = P(Z), gilt: $P(Z \leq \frac{a-\mu}{\sigma})$
        \item  $\Phi$ bzw. Z aus Normalverteilungstabelle \newline auslesen:
              \newline $\Phi(-\frac{a-\mu}{\sigma}) = 1-\Phi(\frac{a-\mu}{\sigma})$
    \end{enumerate}
\end{minipage}%
\begin{minipage}{0.5\textwidth}
    \subsubsection{Binomialverteilung}
    \begin{itemize}
        \item Zwei mögliche Ausgänge $A$ und $\bar{A}$
        \item Wahrscheinlichkeit $p=P(A)$
        \item X = Anzahl eintreten von A
        \item n = Anzahl Durchführungen
    \end{itemize}
    $$P(X=k) = \binom{n}{k}p^k(1-p)^n-k$$
    $$E(X) = np \qquad var(X) = np(1-p)$$
    \subsubsection*{Vergleich Normalverteilung}
    Wenn X Binomialverteilt ist, gilt approximativ:
    $$P(a<X\leq b)$$
    {\tiny Wahrscheinlichkeit das X zwischen a und b liegt}
    $$= P(\frac{a-np}{\sqrt{np(1-p)}}< \frac{X-np}{\sqrt{np(1-p)}} \leq \frac{b-np}{\sqrt{np(1-p)}})$$
    $$= \Phi(\frac{b-np}{\sqrt{np(1-p)}})- \Phi\frac{a-np}{\sqrt{np(1-p)}}$$
    {\tiny $\Phi$ = Normalvereilunsfunktion, änhliches vorgehen wie bei Normalverteilung}

\end{minipage}
\\[20pt]
\begin{minipage}{0.5\textwidth}
    \subsection{Hypergeometrische Verteilung}
    \resizebox{200pt}{!}{%

        \begin{tikzpicture}
            \node[ellipse,
                draw = black,
                minimum width = 6cm,
                minimum height = 3cm] (e) at (2,2) {};
            \fill [fill=green!20!white]
            (e.240) node{} -- (e.120) node{} -- (e.130) node{} -- (e.140) node{}
            -- (e.150) node{} -- (e.160) node{} -- (e.170) node{} -- (e.180) node{}
            -- (e.190) node{} -- (e.200) node{} -- (e.210) node{} -- (e.220) node{}
            -- (e.230) node{} -- (e.240) node{};
            \node[ellipse,
                draw = black,
                line width=0.1mm,
                minimum width = 6cm,
                minimum height = 3cm] at (2,2) {};
            \filldraw[red!30!white]
            (1.7,2) ellipse (1.2cm and 0.75cm);
            \draw[red]
            (1.7,2) ellipse (1.2cm and 0.75cm);
            \draw[line width=0.1mm] (e.120) to (e.240);

            \node[red] at  (1.7, 2.9) {$n$};
            \node[green!80!black] at (0,2.2) {$M$};
            \node[black] at (3,3.7) {$N$};
            \node[green!80!black] at (0.8,2) {$m$};
            \node[black] at (2,2) {$n-m$};
            \node[black] at (3.8,2.4) {$N-M$};
        \end{tikzpicture}
    }

\end{minipage}%
\begin{minipage}{0.5\textwidth}
    \begin{tabular}{clc}
        n aus N auswähbar:        & $\binom{N}{n}$     \\[5pt]
        m Treffer aus M:          & $\binom{M}{m}$     \\[5pt]
        n-m Nichttreffer aus N-M: & $\binom{N-M}{n-m}$
    \end{tabular}
    \\[10pt]
    \textbf{Wahrscheinlichkeit:}
    $$P(X=m) = \frac{\binom{M}{m}\binom{N-M}{n-m}}{\binom{N}{n}}$$
\end{minipage}
\textbf{{Weitere Verteilungen Siehe Anhang}}
\subsection{Ergänzungen für weitere Verteilungen}
\subsubsection{Potenzverteilung}
Gini Koeffizient: $\alpha = \frac{\lambda -2}{\lambda -1}$
\newline Allgemein: Y\% hat X\%  $ \rightarrow X / Y \rightarrow \lambda = \frac{\log{X\%}}{\log{Y\%}}$
\newline zB. für 80/20 Regel $\lambda = \frac{\log{0.8}}{\log{0.2}}$

\newpage
\section{Nullhypothesentest}

\begin{minipage}{0.49\textwidth}
    \subsection{allgemeines Vorgehen}
    \begin{enumerate}
        \item Hypothese $H_0$ aufstellen die der Test
              \newline wiederlegen soll
        \item Festlegung Irrtumswahrscheinlichkeit $\alpha$,
              \newline daraus folgt: $p = 1-\alpha$
              \newline übliche Werte: 0.05 (Signifikant)
              \newline  0.01, 0.001 (hoch Signifikant)
        \item Testgrösse X, Bestimmen der Wahrscheinlichkeitsverteilung (siehe nächster Abschnitt)
              (\item Parameter der Verteilung schätzen)
        \item Bestimmen der Schranken $x_{krit}$
              \begin{itemize}
                  \item Einseitiger Test $P(X> x_{krit})= \alpha$
                  \item Zweiseitiger Test $P(|X| > x_{krit}) = \frac{\alpha}{2}$
              \end{itemize}
        \item Tabellen zur Hilfe nehmen
        \item Falls Messungen ergeben $X > x_{krit}$
              \newline $\rightarrow$ Hypothese falsch mit Wahrscheinlichkeit $p$
    \end{enumerate}
\end{minipage}%
\begin{minipage}{0.49\textwidth}
    \subsection{Diskrete Verteilung}
    \textbf{$\chi^2$-Test}
    \begin{enumerate}
        \item Daten erfassen, Bedingung: $n_{i} \geq 5 \forall i$
        \item Irrtumswahrscheinlichkeit $\alpha$ bestimmen,
              \newline gängige Werte: 0.1, 0.05, 0.01
        \item Freiheitsgrade bestimmen ($i = k-1$)
        \item Diskrepanz berechenen mit Tabelle im Anhang
        \item Kritische Diskrepanz $D_{krit}$ mithilfe von $\chi^2$-Tabelle in Anhang bestimmen
        \item Hypothese beurteilen:
              \begin{itemize}
                  \item Wahrscheinlichkeit $p= 1- \alpha$
                  \item $D \geq D_{krit}$ Hypothese unwahrscheinlich
                  \item $D < D_{krit}$ Hypothese nicht wiederlegbar
                  \item $D \leq D_{krit}$  mit $p= \alpha$
                        \newline Daten möglicherweise künstlich hergestellt
              \end{itemize}
    \end{enumerate}
\end{minipage}
\\[30pt]

\begin{minipage}{0.49\textwidth}
    \subsection{Test einer Verteilung}
    \textbf{Kolmogorov-Smirnov Test}
    \begin{enumerate}
        \item Hypothese: $X$ hat Verteilungsfunktion $F_X$
        \item Werte $x_1$ bis $X_n$ aufsteigend sortieren
        \item $K_n^{\pm}$ berechenen mit Tabelle im Anhang
              \newline $K_n^+ = \sqrt{n} \cdot max(\frac{i}{n}-F_X(x_i))$
              \newline $K_n^- = \sqrt{n} \cdot max(F_X(x_i) - \frac{i-1}{m})$
        \item $t_{n, 1-\alpha}$ und $t_{n, \alpha}$ aus K-Tabelle herauslesen
        \item Wenn $K_n^+ > T_{n, 1-\alpha}$ oder $K_n^- < t_{n, \alpha}$
              \newline $\rightarrow$ Hypothese verwerfen
    \end{enumerate}
\end{minipage}
\begin{minipage}{0.49\textwidth}

    \subsection{Vergleichen von Mittelwerten}
    \textbf{t-Test}
    \begin{enumerate}
        \item Hypothese: $E(X) = E(Y)$
        \item Daten bestimmen:
              \newline $n$ Messungen von $x_i \rightarrow \bar{X} = \frac{X_1 + \dots + X_n}{n}$
              \newline $m$ Messungen von $y_i \rightarrow \bar{Y} = \frac{Y_1 + \dots + Y_m}{m}$
              \newline $S^2_x = \frac{1}{n-1} \cdot \sum_{i=1}^{n}(X_i-\bar{X})^2$
              \newline $S^2_y = \frac{1}{m-1} \cdot \sum_{i=1}^{m}(X_i-\bar{Y})^2$
        \item Testgrösse:
              \newline $T = \frac{\bar{X} - \bar{Y}}{\sqrt{(n-1)
                          \cdot S^2_x + (m-1)\cdot S^2_y}}\cdot \sqrt{\frac{nm(n+m-2)}{n+m}}$

        \item Kritischer Wert $t_{krit}$ aus t-Tabelle auslesen
              \newline $k=n + m - 2$ und $p = 1- \alpha$
        \item Wenn $T>t_{krit}$, Hypothese verwerfen
    \end{enumerate}
\end{minipage}





%%%%%%%%%%%%%%%%%%%%%%%%%%%%%%%%%%%%%%%%%%%%%%%%%%%%%%
%Anhang
%%%%%%%%%%%%%%%%%%%%%%%%%%%%%%%%%%%%%%%%%%%%%%%%%%%%%%
\newpage
\section{Anhang}
\section{Wahl der Verteilung}
\begin{tabular}{|l|p{10cm}|}
    \hline
    Gleichverteilung             & Alle möglichen Werte haben die gleiche Wahrscheinlichkeit                                         \\
    \hline
    Exponentialverteilung        & Dauer von zuf ̈alligen Zeitintervallen ohne Ged ̈achtnis                                            \\
    \hline
    Normalverteilung             & Viele kleine, unabh ̈angige Zufallsprozesse sammeln sich zu einer normalverteilten Zufallsvariable \\
    \hline
    Binomialverteilung           & Experiment mit zwei Ausgängen                                                                     \\
    \hline
    Hypergeometrische Verteilung & Wahrscheinlichkeit, dass in einer Stichprobe, Elemente einer Teilmenge zu finden sind             \\
    \hline
    Poissonverteilung            & Häufigkeiten seltener Ereignisse                                                                  \\
    \hline
    t-Verteilung                 & Varianz aus Stichproben geschätzt (für grosse $n$ kann Normalverteilung verwendet werden)         \\
    \hline
    $\chi^2$-Verteilung          & Test ob eine Zufallsvariable Normalverteilt ist. Anwendung bei Hypothesentests                    \\
    \hline
\end{tabular}
\subsection{Formelzeichen (Auswahl)}
\begin{tabular}{ll}


    $\bar{x} = M_n$         & Mittelwert (i.d.R. Arithmetisches Mittel)                         \\
    $G$                     & Geometrisches Mittel                                              \\
    $var(\dots) = \sigma^2$ & Varianz von \dots                                                 \\
    $\sigma$                & Standardabweichung                                                \\
    $med(\dots)$            & Median von \dots                                                  \\
    $P(\dots) = p $         & Wahrscheinlichkeit für Eintreffen von Ereignis \dots              \\
    $E(\dots) = \mu$        & Erwartungswert von Ereignis \dots                                 \\
    $F(\dots)$              & Verteilungsfunktion von Ereigniss \dots                           \\
    $\varphi(\dots)$        & Wahrscheinlichkeitsverteilung von Ereignis \dots                  \\
    $\alpha$                & Irrtumswahrscheinlichkeit (Gini-Koeffizient bei Potenzverteilung) \\
    $S$                     & Diskrepanz
\end{tabular}

\newpage
\subsection{Taschenrechner}
\begin{itemize}
    \item Modell: TI-nspire CX-II T CAS
    \item Parameter in eckigen Klammern sind optionale Parameter
    \item Parameter mit Komma trennen
    \item $\{\dots\}$ definiert eine Liste, Fett-gedruckte Buchstaben (\textbf{A}) Matrizen
    \item Liste kann mit "ctrl" und ")" eigegeben werden, Elemente mit Komma trennen
    \item Kurzbefehle: Menü-Taste Drücken, Zahlenfolge eintippen
\end{itemize}
\begin{tabular}{|p{5cm}|c|p{10cm}|}
    \hline
    \rowcolor{Gray}
    \textbf{Funktion}                & \textbf{Kurzbefehl}              & \textbf{Beschreibung}                                                                                                    \\
    \hline
    sum(\{...\})                     & -                                & Berechnet die \textbf{Summe} der Elemente einer Liste.                                                                   \\
    \hline
    mean(\{...\} , [\{...\}])        & $6 \rightarrow 3 \rightarrow 3$  & Berechnet das \textbf{arithmetische Mittel} der Liste.
    \newline Zweite Liste = Gewichtung der Elemente                                                                                                                                                \\
    \hline
    mean(\textbf{A} , [\textbf{B}])  & $6 \rightarrow 3 \rightarrow 3$  & Gibt einen Zeilenvektor mit den \textbf{arithmetischen Mitteln}
    \newline der Spalten zurück.
    \newline zweite Matrix B = Gewichtung Elemente von A                                                                                                                                           \\
    \hline
    median(\{...\})                  & $6 \rightarrow 3 \rightarrow 4$  & Berechnet den \textbf{Median} der Elemente der Liste.                                                                    \\
    \hline
    median(\textbf{A})               & $6 \rightarrow 3 \rightarrow 4$  & Gibt einen Zeilenvektor mit den \textbf{Medianwerten} der Spalten zurück.                                                \\
    \hline
    stDevPop(\{...\})                & $6 \rightarrow 3 \rightarrow 9 $ & Berechnet die \textbf{Standardabweichung} $\sigma$ der Liste                                                             \\
    \hline
    varPop(\{...\})                  & $6 \rightarrow 3 \rightarrow A$  & Berechnet die \textbf{Varianz} $\sigma$ der List                                                                         \\
    \hline
    nCr(n, k)                        & $5 \rightarrow 3$                & \textbf{Binomialkoeffizient} $\binom{n}{k}$, Anzahl möglichkeiten k Elemente aus n auszuwählen. \newline \textbf{Kombination}     \\
    \hline
    nPr(n, k)                        & $5 \rightarrow 2$                & Anzahl Möglichkeiten unter Berücksichtigung der Reihenfolge
    \newline k Elemente aus n auszuwählen. \newline \textbf{Permutation}                                                                                                                                    \\
    \hline
    OneVar(L1, [L2], [L3], [L4])     & -                                & Berechnet \textbf{Statistiken} der Liste L1.
    \newline Optionale Parameter:
    \newline[L2]: Häufigkeit, [L3]: Klassencodes, [L4]: Klassenliste                                                                                                                               \\
    \hline
    stat.results                     & $6 \rightarrow 2$                & Zeigt ergebnisse aus letzte Statistik-Berechnung. Anstelle von "result" kann auch auf einzelne Werte zugegriffen werden. \\
    \hline
    Lineare Regression \newline
    {\tiny nur via Eingabeassistent} & $6 \rightarrow 7 \rightarrow A$  & Berechnet Diverse Werte für die \textbf{Lineare Regression}. Werte müssen als Liste eingegeben werden(\{...\}).
    \newline \textbf{Auchtung:} Parameter $a$ + $b$ sind vertauscht.                                                                                                                               \\
    \hline
\end{tabular}

\newpage
\section{Tabellen}
\input{include/NormalverteilungsTabelle.tex}
\subsection{t-Tabelle}
$k=n + m - 2$ und $p = 1- \alpha$ \\
\begin{tabular}{|g|c|c|c|c|c|c|c|}
    \rowcolor{Gray}
    \hline
    k\textbackslash p  & 0.75   & 0.8    & 0.9    & 0.95   & 0.975   & 0.99    & 0.995   \\
    \hline
    1   & 1.0000 & 1.3764 & 3.0777 & 6.3138 & 12.7062 & 31.8205 & 63.6567 \\
    \rowcolor{LightGray}
    2   & 0.8165 & 1.0607 & 1.8856 & 2.9200 & 4.3027  & 6.9646  & 9.9248  \\
    3   & 0.7649 & 0.9785 & 1.6377 & 2.3534 & 3.1824  & 4.5407  & 5.8409  \\
    \rowcolor{LightGray}
    4   & 0.7407 & 0.9410 & 1.5332 & 2.1318 & 2.7764  & 3.7469  & 4.6041  \\
    5   & 0.7267 & 0.9195 & 1.4759 & 2.0150 & 2.5706  & 3.3649  & 4.0321  \\
    \rowcolor{LightGray}
    6   & 0.7176 & 0.9057 & 1.4398 & 1.9432 & 2.4469  & 3.1427  & 3.7074  \\
    7   & 0.7111 & 0.8960 & 1.4149 & 1.8946 & 2.3646  & 2.9980  & 3.4995  \\
    \rowcolor{LightGray}
    8   & 0.7064 & 0.8889 & 1.3968 & 1.8595 & 2.3060  & 2.8965  & 3.3554  \\
    9   & 0.7027 & 0.8834 & 1.3830 & 1.8331 & 2.2622  & 2.8214  & 3.2498  \\
    \rowcolor{LightGray}
    10  & 0.6998 & 0.8791 & 1.3722 & 1.8125 & 2.2281  & 2.7638  & 3.1693  \\
    100 & 0.6770 & 0.8452 & 1.2901 & 1.6602 & 1.9840  & 2.3642  & 2.6259  \\
    \rowcolor{LightGray}
    $12^3$ & 0.6747 & 0.8420 & 1.2824 & 1.6464 & 1.9623  & 2.3301  & 2.5808  \\
    $10^4$ & 0.6745 & 0.8417 & 1.2816 & 1.6450 & 1.9602  & 2.3267  & 2.5763  \\
    \rowcolor{LightGray}
    $10^5$ & 0.6745 & 0.8416 & 1.2816 & 1.6449 & 1.9600  & 2.3264  & 2.5759  \\
    $10^6$ & 0.6745 & 0.8416 & 1.2816 & 1.6449 & 1.9600  & 2.3264  & 2.5758  \\
    \hline
\end{tabular}
\input{include/QuantilenKolmogorovTest.tex}
\input{include/QuantilenXQuadratVerteilung.tex}




\section{Verteilungsdatenblätter ©	Prof. Dr. Andreas Müller}
\includepdf[pages=1-11,landscape, scale=0.95,nup=1x2]{include/VerteilungsDatenblätterByAndreasMueller.pdf}

\section{Berechnungstabellen}
\subsection{Berechnungstabelle Lineare Regression}

\begin{tabular}{|g|p{1cm} p{2cm}|p{2cm} p{3cm}| p{4cm}|}
    \rowcolor{Gray}
    \hline
    $i$                       & $x_i$   & $x_i^2$   & $y_i$   & $y_i^2$   & $x_iy_i$  \\
    \hline
    1                         &         &           &         &           &           \\[15pt]
    \hline
    2                         &         &           &         &           &           \\[15pt]
    \hline
    3                         &         &           &         &           &           \\[15pt]
    \hline
    4                         &         &           &         &           &           \\[15pt]
    \hline
    5                         &         &           &         &           &           \\[15pt]
    \hline
    6                         &         &           &         &           &           \\[15pt]
    \hline
    7                         &         &           &         &           &           \\[15pt]
    \hline
    8                         &         &           &         &           &           \\[15pt]
    \hline
    9                         &         &           &         &           &           \\[15pt]
    \hline
    10                        &         &           &         &           &           \\[15pt]
    \hline
    11                        &         &           &         &           &           \\[15pt]
    \hline
    12                        &         &           &         &           &           \\[15pt]
    \hline
    13                        &         &           &         &           &           \\[15pt]
    \hline
    14                        &         &           &         &           &           \\[15pt]
    \hline
    15                        &         &           &         &           &           \\[15pt]
    \hline
    16                        &         &           &         &           &           \\[15pt]
    \hline
    17                        &         &           &         &           &           \\[15pt]
    \hline
    18                        &         &           &         &           &           \\[15pt]
    \hline
    19                        &         &           &         &           &           \\[15pt]
    \hline
    20                        &         &           &         &           &           \\[15pt]
    \hline
    {\scriptsize $\sum\dots$} & $x_i =$ & $x_i^2 =$ & $y_i =$ & $y_i^2 =$ & $x_iy_i=$ \\[15pt]
    \hline
    {\scriptsize $E$}         & $E(x)=$  & $E(x^2)=$  & $E(y)=$  & $E(y^2)=$  & $E(xy)=$   \\[15pt]
    \hline
    & $Var(x)=$ & & $Var(y)=$ & & $cov(x,y)=$ \\[15pt]
    \hline
\end{tabular}

\newpage\null\thispagestyle{empty}\addtocounter{page}{-1}\newpage %Inserts an empty Page without adding a page-number

\subsection{Berechnungstabelle Diskrepanz (Χ²-Test)}

\begin{tabular}{| g | p{2.5cm} | p{1.5cm} | p{1.5cm} | p{2.5cm} | p{2.5cm} | p{2.5cm} |}
    \rowcolor{Gray}
    \hline
    $i$                 & Beschreibung & $n_i$ & $p_i = \frac{1}{i}$ & $n \cdot p_i$ & $n_i - n \cdot p_i$ & $\frac{(n_i - n \cdot p_i)^2 }{n \cdot p_i}$ \\
    \hline
    1                   &              &       &       &               &                      &                                              \\[10pt]
    \hline
    2                   &              &       &       &               &                      &                                              \\[10pt]
    \hline
    3                   &              &       &       &               &                      &                                              \\[10pt]
    \hline
    4                   &              &       &       &               &                      &                                              \\[10pt]
    \hline
    5                   &              &       &       &               &                      &                                              \\[10pt]
    \hline
    6                   &              &       &       &               &                      &                                              \\[10pt]
    \hline
    7                   &              &       &       &               &                      &                                              \\[10pt]
    \hline
    8                   &              &       &       &               &                      &                                              \\[10pt]
    \hline
    9                   &              &       &       &               &                      &                                              \\[10pt]
    \hline
    10                  &              &       &       &               &                      &                                              \\[10pt]
    \hline
    {\scriptsize Total} &              &$n=$       &       &               &                      &$D=$                                              \\[10pt]
    \hline
\end{tabular}
\subsection{Kolmogorov-Smirnov-Test}

\begin{tabular}{| g | p{3.5cm} | p{3.5cm} | p{3.5cm} | p{3.5cm} | }
    \rowcolor{Gray}
    \hline
    $i$ & $x_i$ (Grösse nach) & $F_X(x_i) $ & $\frac{i}{n}$ & $\frac{i-1}{n}$ \\
    \hline
    1   &                     &             &               &               \\[10pt]
    \hline
    2   &                     &             &               &               \\[10pt]
    \hline
    3   &                     &             &               &               \\[10pt]
    \hline
    4   &                     &             &               &               \\[10pt]
    \hline
    5   &                     &             &               &               \\[10pt]
    \hline
    6   &                     &             &               &               \\[10pt]
    \hline
    7   &                     &             &               &               \\[10pt]
    \hline
    8   &                     &             &               &               \\[10pt]
    \hline
    9   &                     &             &               &               \\[10pt]
    \hline
    10  &                     &             &               &               \\[10pt]
    \hline
\end{tabular}
\end{document}
