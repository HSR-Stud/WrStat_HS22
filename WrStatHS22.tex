% !TeX program = xelatex
% !TeX encoding = utf8
% !TeX root = WrStatHS22.tex

%% TODO: publish to CTAN
\documentclass[margin=normal]{tex/hsrzf}

%%%%%%%%%%%%%%%%%%%%%%%%%%%%%%%%%%%%%%%%%%%%%%%%%%%
% Packages

%% TODO: publish to CTAN
\usepackage{tex/hsrstud}

%% Language configuration
\usepackage{polyglossia}
\usepackage{tikz}
\setdefaultlanguage[variant=swiss]{german}

%% License configuration
\usepackage[
    type={CC},
    modifier={by-nc-sa},
    version={4.0},
    lang={german},
]{doclicense}

%%%%%%%%%%%%%%%%%%%%%%%%%%%%%%%%%%%%%%%%%%%%%%%%%%%
% Metadata

\course{Elektrotechnik}
\module{WrStat}
\semester{Herbstsemester 2022}

\authoremail{joel.leirer@ost.ch}
\author{\textsl{Joël Leirer} -- \texttt{\theauthoremail}}

% did someone help you with this work?
\contributors{

}

\title{\texttt{\themodule} Zusammenfassung}
\date{\thesemester}

%%%%%%%%%%%%%%%%%%%%%%%%%%%%%%%%%%%%%%%%%%%%%%%%%%%
% Document

\begin{document}

% use roman numberals for introductiory pages
\pagenumbering{roman}

\maketitle

% \begin{abstract}
% \end{abstract}

% show the names of the people who contributed to this document.
% \section*{Contributors}
% \thecontributors


\section*{Lizenz}
\doclicenseThis

\clearpage

\tableofcontents

% actual content
\clearpage
\setcounter{page}{1}
\pagenumbering{arabic}

\section{Kombinatorik}
\subsection{Zählregeln}
\begin{tabular}{c c c}
    Disjunktive Vereinigung & Schnittmengen & Produkt \\
    \begin{tikzpicture}
        \fill[red!30!white]   (0,0) circle (0.8);
        \fill[green!30!white] (1.7,0) circle (0.8);      
        \node at (0,0)    {$A$};
        \node at (1.7,0)   {$B$};
    \end{tikzpicture}
    &
    \begin{tikzpicture}
        \begin{scope}[blend group = soft light]
            \fill[red!30!white]   ( 90:0.8) circle (0.8);
            \fill[green!30!white] (180:0.8) circle (0.8);
        \end{scope}
        \node at ( 90:0.8)  {$A$};
        \node at (180:0.8)   {$B$};
    \end{tikzpicture}
    &
    \begin{tikzpicture}
        \fill[black!30!white] (0.3,-0.3) rectangle (1.6,-1.6);
        \fill[red!30!white]  (0.3,0) rectangle (1.6,-0.25);
        \fill[green!30!white](0,-0.3) rectangle (0.25,-1.6);
        \node at (0.95, 0.2) {$A$};
        \node at (-0.2, -0.95) {$B$};
        \node at (0.95, -0.95) {$ A \times B $};
    \end{tikzpicture}
    \\\\
    $|A \cup B| = |A| + |B| $  & 
    $|A \cup B| = |A| + |B| - |A \cap B| $ & 
    $|A \times B| = |A| \cdot |B| $     
\end{tabular}
\subsection{Kombinatorik - Regeln}  
\begin{tabular}{|l| p{10cm}|}
    \hline Anordnung - Permutation & 
    auf wie viele Arten kann man $n$ Objekte anordnen? \newline
    $ n \cdot (n-1) \cdot (n-2) \cdots \cdot 1 = n! $ \\
    \hline Auswahl - Kombination & 
    Auf wie viele Arten kann man k aus n Auswählen? \newline
    Auswahl: $ n \cdot (n-1) \cdot \cdots \cdot (n-(k+1)) = \frac{n!}{(n-k)!}$
    Anordnung der Auswahl:  $k \cdots (k-1) \cdot \cdots \cdot = k!$
    Insgesamt: $ C^n_k = \frac{n!}{k!(n-k)!} = \binom{n}{k} $ \\
    \hline Variation & 
    Perlenkettenproblem:
    auf wie viele Arten kann man eine Perlenkette der Länge k aus n Farben herstellen? \newline
    $V_n,k = n \cdot n^k-1 = n^k$\\
    \hline Rekursion &
    \\ /TODO:
    \hline Erzeugende Funktion &
    \\ /TODO:
    \hline
\end{tabular}

\section{Ereignisse und Wahrscheinlichkeit}

\end{document}
